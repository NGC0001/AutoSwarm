\section{Conclusion}

In this thesis, a hierarchical UAV swarm algorithm is proposed and demonstrated.
The algorithm enables large autonomous swarms
to carry out complex tasks effectively and robustly.
The key idea of the algorithm is to dynamically organise a group of UAVs into a tree structure.
Tasks are divided and executed layer by layer down the tree.
Since messages are mainly sent between each child-parent node pair,
there is no need for many-to-many communication,
improving the flexibility of the swarm and reducing the communication overhead.
Tasks are handled in a divide-and-conquer way,
so the computational overhead is reduced compared to fully distributed algorithms.
When the depth of the tree grows, the number of nodes grows exponentially,
thus large swarm sizes are supported.
The tree structure of the swarm is dynamically organised,
which means the structure can be repaired if any UAV fails,
therefore the structure is robust.

To design and implement the algorithm in detail,
an application scenario is conceived in chapter \ref{chap_problem},
where the task for a swarm is to form designated shapes.
Based on the scenario, the swarm algorithm is developed step by step in chapter \ref{chap_design}.
Then the algorithm is implemented in Rust in chapter \ref{chap_impl}.
A simulation bed is built to carry out simulations in chapter \ref{chap_sim}.

Two simulation cases are run, the simple line task and the letter task.
The results show that the algorithm performs as expected.
In both cases, the tree structure is organised and the designated shapes are formed.
Therefore, the feasibility of the algorithm is verified.
In the letter task, the swarm comprises of 50 UAVs,
showing that the algorithm is suitable for large swarm sizes.

\section{Future Work}

Although the main goal of this thesis is accomplished,
there is still a significant need for further improvement.
\begin{itemize}
    \item Solve the problems listed in section \ref{sec:eval},
          so as to make the algorithm more applicable to real drone shows.
    \item Carry out simulations to
          compare the developed algorithm with other types of swarm algorithms,
          and verify the advantages and disadvantages of the developed algorithm.
    \item Adapt the software, deploy it to real drones, and perform real experiments.
\end{itemize}

Besides, based on the results of this thesis,
there are some interesting directions for future research.
\begin{itemize}
    \item Complex hierarchical swarm structure.
          A hierarchical swarm structure does not have to be a simple tree.
          For example, UAVs can be divided into small groups,
          and these groups are organised into a tree.
          Inside each group, UAVs are equal and fully distributed algorithms are run.
          While between the groups, hierarchical algorithms are run.
          This tree-of-group structure may be more robust than a simple tree.
    \item Tree adaption according to task requirements.
          A swarm will be more powerful
          if it is able to adjust its tree structure for each specific task.
          For example, if a node with 10 descendant nodes has two sub-tasks,
          which require 4 UAVs and 6 UAVs respectively,
          then the node can arrange its descendant nodes into two sub-swarms of size 4 and size 6.
    \item Node management with machine learning methods.
          In this thesis, a state machine is used to manage node status and task status.
          However, for more complex tasks and more complex swarm structures,
          the number of states grows quickly and becomes unmanageable.
          Machine learning methods may help in these situations.
    \item Heterogeneous hierarchical swarm.
          UAVs may be heterogeneous.
          For instance, they may have different communication ranges.
          Organisation and coordination of such swarms are much more complicated.
\end{itemize}