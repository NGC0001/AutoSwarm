Unmanned aerial vehicles (UAVs) have seen more and more application in personal life,
commercial business, industries, and military affairs.
Recently with the advance of technology, scenarios emerge where a group of UAVs
carry out an assigned task in a coordinated way \parencite{Dimakos2024, Javed2024, Ouyang2023}.
Such UAV groups are called UAV swarms.
Swarms are much more complex than individual UAVs,
and application of swarms is still in its infancy.

To fully explore and utilise UAV swarms, a lot of researchers have investigated algorithms
related to the communication, control, and task management of swarms
\parencite{Javaid2023, Zhou2020}.
Such algorithms can be roughly divided into two categories,
\textit{centralised algorithms} where a leader UAV is in charge of all the other UAVs,
and \textit{decentralised algorithms} where all UAVs are treated equally.
Decentralised algorithms can further be divided into two types,
\textit{negotiation-based algorithms} where each UAV negotiates with other UAVs to make decisions,
and \textit{reaction-based algorithms} where UAVs achieve collective behaviour through
observing other UAVs and reacting according to preset rules.

Centralised algorithms are simple,
but are prone to single points of failure induced by the leader.
Reaction-based algorithms are robust,
but are not suitable for complex tasks where elaborate coordination among UAVs is necessary.
Negotiation-based algorithms are powerful,
but entails a large amount of computation and all-to-all communication during negotiation,
effectively limiting the size of the swarm.

The \textbf{aim} of this thesis is
to \textit{find a robust algorithm suitable for large swarms and complex tasks}.
The proposed \textbf{solution} is
a \textit{hierarchical swarm} where UAVs dynamically organise themselves into a tree structure.
Instead of all-to-all communication, messages are mainly sent between UAVs
whose corresponding tree nodes are directly connected.
Tasks are managed in a divide-and-conquer way layer by layer down the tree.

To achieve this aim, the thesis below presents the following contributions:
\begin{enumerate}
    \item Firstly a \textit{swarm application scenario} is conceived (see chapter \ref{chap_problem}).
    \item Then based on the scenario,
          the \textit{algorithm is proposed} (see section \ref{sec:alg_design}),
          and the \textit{software architecture} that embodies this algorithm is \textit{designed}
           (see section \ref{sec:sft_arch}).
    \item Thirdly, a detailed \textit{implementation is built}
          to demonstrate the viability of the proposed solution (see chapter \ref{chap_impl}).
    \item Lastly,
          \textit{simulation is carried out to evaluate the algorithm} (see chapter \ref{chap_sim}).
          The results verified that the algorithm is suitable for large swarms and complex tasks. 
\end{enumerate}