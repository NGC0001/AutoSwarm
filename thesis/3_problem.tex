To develop and demonstrate a swarm algorithm,
an application scenario is needed.
Inspired by drone shows, where drones form various interesting shapes,
in this thesis, the task for a swarm is to form a designated geometric shape.
Since the purpose is focused on swarm control and coordination, many other aspects,
such as aerodynamics, obstacle avoidance, and power or fuel management,
are ignored or simplified.
The problem is formulated below.

\section{UAV Properties}

At time point $t = 0$, a set of homogeneous UAVs
$W = \{U_i | i = 0, 1, \ldots, N_u-1; N_u > 0\}$
are located at initial positions $\bm{P}_i(t=0) = (x_{i0}, y_{i0}, z_{i0})$.
They are already airborne. The taking-off process is not considered.

Each UAV $U_i$ has a unique ID $id_i$.
The ID set has total order, which means IDs are comparable.

GPS receivers are installed, so UAVs can read
their positions $\bm{P}_i(t) = (x_i, y_i, z_i)$.
The distance between two UAVs $U_i$ and $U_j$ is
\begin{equation}
d_{ij}(t) = d_{ji}(t) = ||\bm{P}_i(t) - \bm{P}_j(t)||.
\end{equation}
They will clash if $d_{ij} \leqslant 2 r_{rad}$,
where $r_{rad}$ is the radius of a UAV.

UAVs are able to move at arbitrary velocity $\bm{V}_i(t) = (vx_i, vy_i, vz_i)$,
as long as
\begin{equation}
v_i(t) = ||\bm{V}_i(t)|| \leqslant v_{max},
\end{equation}
where $v_{max}$ is the given maximum speed of the UAVs.
Hovering, i.e., $\bm{V} = \bm{0}$, is allowed.
In real cases, a UAV can only control its engines or rotors.
The actual acceleration and velocity are determined by engine power and aerodynamics.
In this thesis, the flight control is greatly simplified,
and the UAVs are assumed to be able to change their velocity directly.
Besides, it's assumed that the UAVs fly in the same manner in all directions,
without the need to care about attitude,
which is of course not true for any real UAVs.

UAVs do not known the existence of each other at $t = 0$.
UAV $U_i$ can receive all the data sent by UAV $U_j$
if the distance between them is within the given communication range $r_{comm}$,
i.e., $d_{ij} \leqslant r_{comm}$.
Latency and bandwidth problems are not considered.
At $t = 0$, the initial positions of the UAVs ensure that,
any two UAVs are within communication range.

\section{Representation of Geometric Shapes}

A ``shape" here is made up of one or multiple lines.
While a ``line" here is made up of one or multiple connected line segments.
A line segment is defined by two points,
so a line can be defined by a sequence of points.

Let $L$ represent a sequence of $N_p$ different points
$\bm{P}_0, \bm{P}_1, \ldots, \bm{P}_{N_p-1}$,
where $\bm{P}_k = (x_k, y_k, z_k), 0 \leqslant k \leqslant N_p-1$,
and $N_p \geqslant 2$.
$L$ defines $N_p - 1$ connected line segments, which form a ``line".
A point $\bm{P} = (x, y, z)$ is said to be on line $L$
if it falls onto any of the line segments,
that is, there exist an integer $i, 0 \leqslant i < N_p-1$,
and a real number $c, 0 \leqslant c \leqslant 1$, such that
\begin{eqnarray}
    (x - x_i) &=& c(x_{i+1} - x_i), \\
    (y - y_i) &=& c(y_{i+1} - y_i), \\
    (z - z_i) &=& c(z_{i+1} - z_i).
\end{eqnarray}
A shape $S$ is a set of $N_l$ lines $\{L_m | m = 0, 1, \ldots, N_l-1; N_l > 0\}$.
$\bm{P}$ is said to be on $S$ if it is on any of the lines.

Lines defined above can not represent curves.
However, a curve can be approximated by a large amount of short line segments,
in the commonly used way how a circle is approximated by a polygon.

\section{Swarm Tasks}
\label{sec:swm_tsk}

A task $TSK$ contains a unique task ID $tid$, a shape $S$ and a time period $\Delta t$.

Let $W$ be a set of UAVs.
At some time point $t'$,
task $TSK$ is sent by GCS to an arbitrary member of $W$.
The mission for $W$ is,
all the UAVs move onto the shape and stay for $\Delta t$.
That is, the task is said to be successfully executed
if there exists a time point $T, t' \leqslant T < +\infty$,
such that for any time point $t \in [T, T+\Delta t]$ and for any UAV $U \in W$,
$U$ is on $S$.

Collisions shall be avoided.
That is, at any time point $t$, for any two UAVs $U_i$ and $U_j$,
the distance between them shall satisfy $d_{ij} > 2 r_{rad}$.

Additionally, UAVs should be distributed as evenly as possible along the lines.
This means to minimise the mean square root of UAV intervals along the lines.
This is not the primary goal of the task, so it is not formally stated here.
But later in section \ref{sec:tsk_div},
an algorithm will be implemented to address this goal.